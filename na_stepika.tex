\documentclass{article}
\usepackage{graphicx}
\usepackage{lipsum}

\title{Stepik}
\author{Paulina Krajnik}
\date{\today}

\begin{document}
\maketitle

\begin{abstract}
    \lipsum[1-3]
\end{abstract}

\section{Wprowadzenie}
    \lipsum[4-7]

\section{Rozdział 1}
    \subsection{Podrozdział 1.1}
	Równanie kwadratowe ma postać:
	\[ ax^2 + bx + c = 0 \]
	Rozwiazania tego rownania sa dane wzorem kwadratowym:
	\[ x = \frac{-b \pm \sqrt{b^2 - 4ac}}{2a} \]

        \underline{\lipsum[7-12]}
   	 \underline{podkreślony jakieś tekst......}

    \subsection{Podrozdział 1.2}
	\lipsum[12-15]

 	Przykład równania matematycznego: $E=mc^2$.

\section{Rozdział 2}
    \lipsum[15-23]


\section{Rysunki i Tabele}
    \begin{figure}[h]
        \centering
        \includegraphics[width=0.5\textwidth]{rysunek1.png}
        \caption{Podpis mojego rysunku grzybka}
	\textit{Tekst kursywa, jakiś tekst tralalala lalalaaaaaa}
        \label{fig:rys1}
    \end{figure}

\section{Tabelka}

\begin{table}[ht]
  \centering
  \begin{tabular}{|c|c|}
    \hline
    wartosci & inne \\
    \hline
    zapasy & dobrze \\
    \hline
    mozliwe & ll \\
    \hline
     Costamm & costam \\
    \hline
  \end{tabular}
  \caption{moja tabelka}
   \label{tab:tabelka1}
\end{table}

\section{Podsumowanie}

Tekst podsumowujacy artykuł.

Odniesienie do tabeli znajduje sie w Tabeli \ref{tab:tabelka1}.

    \lipsum[1]
  \begin{figure}[h]
  \centering
  \includegraphics[width=0.6\textwidth]{rys2.png}
  \caption{Skomponowany \textbf{rys1}}
  \label{opis: }
\end{figure}

    \begin{table}[h]
        \centering
        \begin{tabular}{|c|c|}
            \hline
            Nagłówek 1 & Nagłówek 2 \\
            \hline
            Wiersz 1.1 & Wiersz 1.2 \\
            Wiersz 2.1 & Wiersz 2.2 \\
            \hline
        \end{tabular}
        \caption{tabelka}
        \label{tab:tabelka2}
    \end{table}

\section{Odniesienia}
    Odniesienie do Rysunku \ref{fig:rys1} oraz Tabeli \ref{tab:tabelka2}.

\section{Podsumowanie}
    \lipsum[5]

\begin{thebibliography}{99}
    \bibitem{Autor1} Autor1. \emph{Tytuł 1.} Wydawnictwo, Rok.
    \bibitem{Autor2} Autor2. \emph{Tytuł 2.} Wydawnictwo, Rok.
\end{thebibliography}

\end{document}

