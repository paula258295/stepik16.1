\documentclass{article}
\usepackage{graphicx}
\usepackage{lipsum}

\title{Stepik}
\author{Paulina Krajnik}
\date{\today}

\begin{document}
\maketitle

\begin{abstract}
    To jest streszczenie pracy.
\end{abstract}

\section{Wprowadzenie}
    \lipsum[1]

\section{Rozdział 1}
    \subsection{Podrozdział 1.1}
        \lipsum[2]
    
    \subsection{Podrozdział 1.2}
        Przykład równania matematycznego: $E=mc^2$.

\section{Rozdział 2}
    \lipsum[3]

\section{Rysunki i Tabele}
    \begin{figure}[h]
        \centering
        \includegraphics[width=0.5\textwidth]{rysunek1.png}
        \caption{Podpis do rysunku 1.}
        \label{fig:rys1}
    \end{figure}

    \lipsum[4]

    \begin{table}[h]
        \centering
        \begin{tabular}{|c|c|}
            \hline
            Nagłówek 1 & Nagłówek 2 \\
            \hline
            Wiersz 1.1 & Wiersz 1.2 \\
            Wiersz 2.1 & Wiersz 2.2 \\
            \hline
        \end{tabular}
        \caption{Tabela 1.}
        \label{tab:tabela1}
    \end{table}

\section{Odniesienia}
    Odniesienie do Rysunku \ref{fig:rys1} oraz Tabeli \ref{tab:tabela1}.

\section{Podsumowanie}
    \lipsum[5]

\begin{thebibliography}{99}
    \bibitem{Autor1} Autor1. \emph{Tytuł 1.} Wydawnictwo, Rok.
    \bibitem{Autor2} Autor2. \emph{Tytuł 2.} Wydawnictwo, Rok.
\end{thebibliography}

\end{document}

